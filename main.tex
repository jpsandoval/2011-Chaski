%%%%%%%%%%%%%
% % % % % % % % % % % % % % % % % % % % % % % % % % % % % % % % % %
%\documentclass[runningheads]{llncs}
%\documentclass[10pt,letterpaper,twocolumn]{article}
\documentclass{sig-alternate}


% packages
\usepackage{xspace}
\usepackage{ifthen}
\usepackage{amsbsy}
\usepackage{amssymb}
\usepackage{balance}
\usepackage{booktabs}
\usepackage{graphicx}
\usepackage{multirow}
\usepackage{needspace}
\usepackage{microtype}
\usepackage{bold-extra}
\usepackage{subfigure}
\usepackage{wrapfig}


% constants
\newcommand{\Title}{Execution Blueprint: An Effective Visual Support to Monitor Software Execution}
\newcommand{\TitleShort}{\Title}
\newcommand{\Authors}{Juan Pablo Sandoval Alcoce, Alexandre Bergel}
\newcommand{\AuthorsShort}{J.P. Sandoval Alcose, A. Bergel}

% references
\usepackage[colorlinks]{hyperref}
\usepackage[all]{hypcap}
\setcounter{tocdepth}{2}
\hypersetup{
	colorlinks=true,
	urlcolor=black,
	linkcolor=black,
	citecolor=black,
	plainpages=false,
	bookmarksopen=true,
	pdfauthor={\Authors},
	pdftitle={\Title}}

\def\chapterautorefname{Chapter}
\def\appendixautorefname{Appendix}
\def\sectionautorefname{Section}
\def\subsectionautorefname{Section}
\def\figureautorefname{Figure}
\def\tableautorefname{Table}
\def\listingautorefname{Listing}

% source code
\usepackage{xcolor}
\usepackage{textcomp}
\usepackage{listings}
\definecolor{source}{gray}{0.9}
\lstset{
	language={},
	% characters
	tabsize=3,
	upquote=true,
	escapechar={!},
	keepspaces=true,
	breaklines=true,
	alsoletter={\#:},
	breakautoindent=true,
	columns=fullflexible,
	showstringspaces=false,
	basicstyle=\footnotesize\sffamily,
	% background
	frame=single,
    framerule=0pt,
	backgroundcolor=\color{source},
	% numbering
	numbersep=5pt,
	numberstyle=\tiny,
	numberfirstline=true,
	% captioning
	captionpos=b,
	% formatting (html)
	moredelim=[is][\textbf]{<b>}{</b>},
	moredelim=[is][\textit]{<i>}{</i>},
	moredelim=[is][\color{red}\uwave]{<u>}{</u>},
	moredelim=[is][\color{red}\sout]{<del>}{</del>},
	moredelim=[is][\color{blue}\underline]{<ins>}{</ins>}}
\newcommand{\ct}{\lstinline[backgroundcolor=\color{white},basicstyle=\footnotesize\ttfamily]}
\newcommand{\lct}[1]{{\small\tt #1}}

% tikz
% \usepackage{tikz}
% \usetikzlibrary{matrix}
% \usetikzlibrary{arrows}
% \usetikzlibrary{external}
% \usetikzlibrary{positioning}
% \usetikzlibrary{shapes.multipart}
% 
% \tikzset{
% 	every picture/.style={semithick},
% 	every text node part/.style={align=center}}

% proof-reading
\usepackage{xcolor}
\usepackage[normalem]{ulem}
\newcommand{\ra}{$\rightarrow$}
\newcommand{\ugh}[1]{\textcolor{red}{\uwave{#1}}} % please rephrase
\newcommand{\ins}[1]{\textcolor{blue}{\uline{#1}}} % please insert
\newcommand{\del}[1]{\textcolor{red}{\sout{#1}}} % please delete
\newcommand{\chg}[2]{\textcolor{red}{\sout{#1}}{\ra}\textcolor{blue}{\uline{#2}}} % please change
\newcommand{\chk}[1]{\textcolor{ForestGreen}{#1}} % changed, please check

% comments \nb{label}{color}{text}
\newboolean{showcomments}
\setboolean{showcomments}{true}
\ifthenelse{\boolean{showcomments}}
	{\newcommand{\nb}[3]{
		{\colorbox{#2}{\bfseries\sffamily\scriptsize\textcolor{white}{#1}}}
		{\textcolor{#2}{\sf\small$\blacktriangleright$\textit{#3}$\blacktriangleleft$}}}
	 \newcommand{\version}{\emph{\scriptsize$-$Id$-$}}}
	{\newcommand{\nb}[2]{}
	 \newcommand{\version}{}}
\newcommand{\rev}[2]{\nb{Reviewer #1}{red}{#2}}
\newcommand{\ab}[1]{\nb{Alexandre}{blue}{#1}}
\newcommand{\vp}[1]{\nb{Vanessa}{orange}{#1}}

% graphics: \fig{position}{percentage-width}{filename}{caption}
\DeclareGraphicsExtensions{.png,.jpg,.pdf,.eps,.gif}
\graphicspath{{figures/}}
\newcommand{\fig}[4]{
	\begin{figure}[#1]
		\centering
		\includegraphics[width=#2\textwidth]{#3}
		\caption{\label{fig:#3}#4}
	\end{figure}}

\newcommand{\largefig}[4]{
	\begin{figure*}[#1]
		\centering
		\includegraphics[width=#2\textwidth]{#3}
		\caption{\label{fig:#3}#4}
	\end{figure*}}
	
\newcommand{\wrapfig}[5]{	
\begin{wrapfigure}{#1}{#2\textwidth}
  \begin{center}
    \includegraphics[width=#3\textwidth]{#4}
  \end{center}
  \caption{\label{fig:#4}#5}
\end{wrapfigure}}

% abbreviations
\newcommand{\ie}{\emph{i.e.,}\xspace}
\newcommand{\eg}{\emph{e.g.,}\xspace}
\newcommand{\etc}{\emph{etc.}\xspace}
\newcommand{\etal}{\emph{et al.}\xspace}

% lists
\newenvironment{bullets}[0]
	{\begin{itemize}}
	{\end{itemize}}

\newcommand{\seclabel}[1]{\label{sec:#1}}
\newcommand{\figlabel}[1]{\label{fig:#1}}
\newcommand{\figref}[1]{Figure~\ref{fig:#1}}
\newcommand{\secref}[1]{Section~\ref{sec:#1}}


%Specialized macros
\newcommand{\hapao}{Hapao\xspace}
\newcommand{\Hapao}{Hapao\xspace}
\pagenumbering{arabic}

\begin{document}

\title{\Title}
%\titlerunning{\TitleShort}

\author{\Authors\\[3mm]
Department of Computer Science (DCC)\\ University of Chile, Santiago, Chile\\[1 ex]
} 
%\authorrunning{\AuthorsShort}

\maketitle

\emph{This paper makes use of colored figures. Though colors are not mandatory for full understanding, we recommend  the use of a colored printout.}

\begin{abstract}
%	What's the problem.

%	Why is the problem a problem?

%	What's the surprising idea?

%	What's the consequence?
\end{abstract}

%: % % % % % % % % % % % % % % % % % % % % % % % % % % % % % % % % %

\section{Introduction}\seclabel{introduction}

%: % % % % % % % % % % % % % % % % % % % % % % % % % % % % % % % % %
\section{Execution Blueprint}\seclabel{executionBlueprint}

\subsection{In a nutshell}

\subsection{Patterns}

%: % % % % % % % % % % % % % % % % % % % % % % % % % % % % % % % % %
\section{Case Study}\seclabel{case study}


\subsection{Mondrian}

\subsection{Execution anomalies}

\subsection{Improvement}


%: % % % % % % % % % % % % % % % % % % % % % % % % % % % % % % % % %
\section{Implementation}\seclabel{implementation}

\subsection{Chaski}


%: % % % % % % % % % % % % % % % % % % % % % % % % % % % % % % % % %
\section{Related work}\seclabel{relatedwork}


\ab{Fill the section above with what is below}
Understanding the behavior of an object oriented application is a difficult task, having a visualization that represents the interaction of objects involved in the execution can help developers gain a better understanding, but the large number of objects involved in execution represents a major challenge.

Chaski is a tool created to take this challenge by visualizing the execution of an object oriented program, showing the interaction between objects through messages exchanged between them, in the order they occur. This visualization helps to: understand the current behavior of the software, find candidates for refactoring and optimization, show dependencies and relations, among others.

The main difficulty of making a dynamic analysis is the large amount of data collected from the execution of code, so as performance and scalability problems.

One of the biggest problems in software maintenance is to understand the behavior of the software, dynamic analysis is necessary to perform this task.

It is difficult to navigate the vast(large) amount of information obtained from the code execution, always trying not to get lost in the visualization. (Browser)


Layout Chaski

The essence of chaski are the messages and the chronological order in which these (messages) occur.

The large-scale sequence of messages makes the screen resolution not enough to show a lot of information,causing the emergence of scrolls that make the graphics difficult to understand.

The layout of chaski allows the user to group the messages by different criteria, such as per package, class, object, method with blue, red, yellow and green colors, respectively, or a combination of these but respecting the hierarchy.

Detecting important messages.

To help mark or highlight the messages you want to analyze, chaski provides a simple mechanism of colored rules of messages, it allows to color a message given a particular criterion. If you have multiple criteria and, in the case that one or more messages meet more than one criterion, these will take the color of the last criterion.

Important message filtering.

So like a lot of objects are created when executing an object-oriented program,  so much larger the number of messages that are sent between objects, making the visualization extremely large. The filters in chaski try to display messages that meet a certain condition allowing a smaller view and most importantly, making  possible to analyze only the messages that interest us.



ANOTHER IDEAS
---------------------------------------------------------------------

An adequate understanding of a software system is an essential task of the software life cycle.

Analyze the code and documentation are needed to understanding the behavior of an object oriented application,  leading to the creation of mind maps that give an overview of behav.

To maintain a software first we must know all software behavior (Ideally) before making any change, you want to get an understanding of the part of the system that you are specifically interested in order to make a change. Chaski allows this type of analysis goal oriented to accelerate the understanding of the software in order to make your goal quickly.

Reusing an object

An object-oriented program may involve a large number of objects, in this context  can be found redundance into objects that can contain identical information, leading a misuse of resources. 
One of the advantages of object-oriented programming is flexivilidad and code reuse. 
How could we achieve a balance between flexivilidad and performance (in terms of resource use).

%: % % % % % % % % % % % % % % % % % % % % % % % % % % % % % % % % %

\section{Conclusion}\seclabel{conclusion}

%\paragraph{Acknowledgment}
% % % % % % % % % % % % % % % % % % % % % % % % % % % % % % % % % %

\bibliographystyle{plain}
\bibliography{scg}

\end{document}

