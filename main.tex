%%%%%%%%%%%%%
% % % % % % % % % % % % % % % % % % % % % % % % % % % % % % % % % %
%\documentclass[runningheads]{llncs}
%\documentclass[10pt,letterpaper,twocolumn]{article}
\documentclass{sig-alternate}


% packages
\usepackage{xspace}
\usepackage{ifthen}
\usepackage{amsbsy}
\usepackage{amssymb}
\usepackage{balance}
\usepackage{booktabs}
\usepackage{graphicx}
\usepackage{multirow}
\usepackage{needspace}
\usepackage{microtype}
\usepackage{bold-extra}
\usepackage{subfigure}
\usepackage{wrapfig}


% constants
\newcommand{\Title}{Execution Blueprint: An Effective Visual Support to Monitor Software Execution}
\newcommand{\TitleShort}{\Title}
\newcommand{\Authors}{Juan Pablo Sandoval Alcocer, Alexandre Bergel}
\newcommand{\AuthorsShort}{J.P. Sandoval Alcocer, A. Bergel}

% references
\usepackage[colorlinks]{hyperref}
\usepackage[all]{hypcap}
\setcounter{tocdepth}{2}
\hypersetup{
	colorlinks=true,
	urlcolor=black,
	linkcolor=black,
	citecolor=black,
	plainpages=false,
	bookmarksopen=true,
	pdfauthor={\Authors},
	pdftitle={\Title}}

\def\chapterautorefname{Chapter}
\def\appendixautorefname{Appendix}
\def\sectionautorefname{Section}
\def\subsectionautorefname{Section}
\def\figureautorefname{Figure}
\def\tableautorefname{Table}
\def\listingautorefname{Listing}

% source code
\usepackage{xcolor}
\usepackage{textcomp}
\usepackage{listings}
\definecolor{source}{gray}{0.9}
\lstset{
	language={},
	% characters
	tabsize=3,
	upquote=true,
	escapechar={!},
	keepspaces=true,
	breaklines=true,
	alsoletter={\#:},
	breakautoindent=true,
	columns=fullflexible,
	showstringspaces=false,
	basicstyle=\footnotesize\sffamily,
	% background
	frame=single,
    framerule=0pt,
	backgroundcolor=\color{source},
	% numbering
	numbersep=5pt,
	numberstyle=\tiny,
	numberfirstline=true,
	% captioning
	captionpos=b,
	% formatting (html)
	moredelim=[is][\textbf]{<b>}{</b>},
	moredelim=[is][\textit]{<i>}{</i>},
	moredelim=[is][\color{red}\uwave]{<u>}{</u>},
	moredelim=[is][\color{red}\sout]{<del>}{</del>},
	moredelim=[is][\color{blue}\underline]{<ins>}{</ins>}}
\newcommand{\ct}{\lstinline[backgroundcolor=\color{white},basicstyle=\footnotesize\ttfamily]}
\newcommand{\lct}[1]{{\small\tt #1}}

% tikz
% \usepackage{tikz}
% \usetikzlibrary{matrix}
% \usetikzlibrary{arrows}
% \usetikzlibrary{external}
% \usetikzlibrary{positioning}
% \usetikzlibrary{shapes.multipart}
% 
% \tikzset{
% 	every picture/.style={semithick},
% 	every text node part/.style={align=center}}

% proof-reading
\usepackage{xcolor}
\usepackage[normalem]{ulem}
\newcommand{\ra}{$\rightarrow$}
\newcommand{\ugh}[1]{\textcolor{red}{\uwave{#1}}} % please rephrase
\newcommand{\ins}[1]{\textcolor{blue}{\uline{#1}}} % please insert
\newcommand{\del}[1]{\textcolor{red}{\sout{#1}}} % please delete
\newcommand{\chg}[2]{\textcolor{red}{\sout{#1}}{\ra}\textcolor{blue}{\uline{#2}}} % please change
\newcommand{\chk}[1]{\textcolor{ForestGreen}{#1}} % changed, please check

% comments \nb{label}{color}{text}
\newboolean{showcomments}
\setboolean{showcomments}{true}
\ifthenelse{\boolean{showcomments}}
	{\newcommand{\nb}[3]{
		{\colorbox{#2}{\bfseries\sffamily\scriptsize\textcolor{white}{#1}}}
		{\textcolor{#2}{\sf\small$\blacktriangleright$\textit{#3}$\blacktriangleleft$}}}
	 \newcommand{\version}{\emph{\scriptsize$-$Id$-$}}}
	{\newcommand{\nb}[2]{}
	 \newcommand{\version}{}}
\newcommand{\rev}[2]{\nb{Reviewer #1}{red}{#2}}
\newcommand{\ab}[1]{\nb{Alexandre}{blue}{#1}}
\newcommand{\vp}[1]{\nb{Vanessa}{orange}{#1}}

% graphics: \fig{position}{percentage-width}{filename}{caption}
\DeclareGraphicsExtensions{.png,.jpg,.pdf,.eps,.gif}
\graphicspath{{figures/}}
\newcommand{\fig}[4]{
	\begin{figure}[#1]
		\centering
		\includegraphics[width=#2\textwidth]{#3}
		\caption{\label{fig:#3}#4}
	\end{figure}}

\newcommand{\largefig}[4]{
	\begin{figure*}[#1]
		\centering
		\includegraphics[width=#2\textwidth]{#3}
		\caption{\label{fig:#3}#4}
	\end{figure*}}
	
\newcommand{\wrapfig}[5]{	
\begin{wrapfigure}{#1}{#2\textwidth}
  \begin{center}
    \includegraphics[width=#3\textwidth]{#4}
  \end{center}
  \caption{\label{fig:#4}#5}
\end{wrapfigure}}

% abbreviations
\newcommand{\ie}{\emph{i.e.,}\xspace}
\newcommand{\eg}{\emph{e.g.,}\xspace}
\newcommand{\etc}{\emph{etc.}\xspace}
\newcommand{\etal}{\emph{et al.}\xspace}

% lists
\newenvironment{bullets}[0]
	{\begin{itemize}}
	{\end{itemize}}

\newcommand{\seclabel}[1]{\label{sec:#1}}
\newcommand{\figlabel}[1]{\label{fig:#1}}
\newcommand{\figref}[1]{Figure~\ref{fig:#1}}
\newcommand{\secref}[1]{Section~\ref{sec:#1}}


%Specialized macros
\newcommand{\hapao}{Hapao\xspace}
\newcommand{\Hapao}{Hapao\xspace}
\pagenumbering{arabic}

\begin{document}

\title{\Title}
%\titlerunning{\TitleShort}

\author{\Authors\\[3mm]
Department of Computer Science (DCC)\\ University of Chile, Santiago, Chile\\[1 ex]
} 
%\authorrunning{\AuthorsShort}

\maketitle

\emph{This paper makes use of colored figures. Though colors are not mandatory for full understanding, we recommend  the use of a colored printout.}

\begin{abstract}
%	What's the problem.
%	Why is the problem a problem?
%	What's the surprising idea?
%	What's the consequence?

Understanding the behavior of software is one essential task of the software life cycle, especially when maintenance \chg{is done to it}{activities have to be performed}. One has to analyze the code, the documentation or any other device before making any changes therefore it becomes a difficult task and time consuming.

Our goal is to show a visualization that represents the interaction of objects involved in the execution which can help developers gain a better understanding \ins{of the software execution}\del{, but}\ins{.} \chg{t}{T}he large number of objects involved in execution represents a major challenge\del{ (Chaski)}.

Chaski is a tool \del{created} to \chg{take this challenge}{address this problem of understanding} by visualizing the execution of an object oriented program\chg{,}{.} \ins{Execution blueprint} \chg{showing}{shows} the interaction between objects through messages exchanged between them, in the order they occur. \ab{Say more about the visualization}. This visualization helps to: understand the current behavior of the software, find candidates for refactoring and optimization, show dependencies and relations, among others.


\end{abstract}

%: % % % % % % % % % % % % % % % % % % % % % % % % % % % % % % % % %

\section{Introduction}\seclabel{introduction}

The use of highly reflective dynamic languages can make seemingly simple questions(how 2 or more packages interact? What kinds are dependent to a certain class? When exactly this method is invoked?)difficult to answer if we just analyze the source code, documentation or any other appliance design.

To answer these questions among others and to improve the process of understanding, it is necessary to make a dynamic analysis, or analysis of data retrieved from the execution of a program.

In this context, we propose a visualization of the execution of an object oriented program, showing the interaction between objects through messages exchanged between them, in the order they occur. While understanding the overall behavior of the system is ideal, this visualization was intended to be used by objectives, in other words, to analyze portions of the application that really interest us to achieve an objective.

The purpose of this visualization is to help: understand the current behavior of the software, find candidates for refactoring and optimization, show dependencies and relations, among others.

%: % % % % % % % % % % % % % % % % % % % % % % % % % % % % % % % % %
\section{Execution Blueprint}\seclabel{executionBlueprint}

\subsection{In a nutshell}

\ab{In this section you already go into details. This has to come afterwards. Start with a cool visualization and says you see in the visualization. Subsequent sections will then go into details.}

\paragraph{Layout}
The essence of chaski are the messages and the chronological order in which these (messages) occur. \ab{We are not talking about Chaski here. Chaski has to be mentioned in the implementation. We need to talk about execution blueprint, the visualization given by Chaski.}

\ab{Give an example of a cool visualization}
\fig{}{0.7}{Chaski22}{What a cool example!}

The large-scale sequence of messages makes the screen resolution not enough to show a lot of information,causing the emergence of scrolls that make the graphics difficult to understand\ab{This is an important point, but this has to come later on in the paper. Maybe in the discussion session}.

The layout of chaski allows the user to group the messages by different criteria, such as per package, class, object, method with blue, red, yellow and green colors, respectively, or a combination of these but respecting the hierarchy.

\paragraph{Coloreando mensajes}
To help mark or highlight the messages you want to analyze, chaski provides a simple mechanism of colored rules of messages, it allows to color a message given a particular criterion. If you have multiple criteria and, in the case that one or more messages meet more than one criterion, these will take the color of the last criterion.

\paragraph{Aplicando filtros}
\ab{Move this paragraph into the scalability section} So like a lot of objects are created when executing an object-oriented program,  so much larger the number of messages that are sent between objects, making the visualization extremely large. The filters in chaski try to display messages that meet a certain condition allowing a smaller view and most importantly, making  possible to analyze only the messages that interest us.

Dependencia de mensajes.


Secuencia de mensajes.

%=========
\subsection{Layout}
\ab{Move the paragraph layout in this new subsection}

%=========
\subsection{Patterns}

\ab{Tell about the patterns you have showed me}


%: % % % % % % % % % % % % % % % % % % % % % % % % % % % % % % % % %
\section{Case Study}\seclabel{case study}

\subsection{Mondrian}

\subsection{Execution anomalies}

\subsection{Improvement}

%: % % % % % % % % % % % % % % % % % % % % % % % % % % % % % % % % %
\section{Managing the scalability}\seclabel{scalability}


%: % % % % % % % % % % % % % % % % % % % % % % % % % % % % % % % % %
\section{Discussion}\seclabel{discussion}


%: % % % % % % % % % % % % % % % % % % % % % % % % % % % % % % % % %
\section{Implementation}\seclabel{implementation}

\subsection{Chaski}


%: % % % % % % % % % % % % % % % % % % % % % % % % % % % % % % % % %
\section{Related work}\seclabel{relatedwork}


%: % % % % % % % % % % % % % % % % % % % % % % % % % % % % % % % % %

\section{Conclusion}\seclabel{conclusion}

%\paragraph{Acknowledgment}
% % % % % % % % % % % % % % % % % % % % % % % % % % % % % % % % % %

\bibliographystyle{plain}
\bibliography{scg}

\end{document}

