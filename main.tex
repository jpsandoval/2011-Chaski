%%%%%%%%%%%%%
% % % % % % % % % % % % % % % % % % % % % % % % % % % % % % % % % %
%\documentclass[runningheads]{llncs}
%\documentclass[10pt,letterpaper,twocolumn]{article}
\documentclass{sig-alternate}


% packages
\usepackage{xspace}
\usepackage{ifthen}
\usepackage{amsbsy}
\usepackage{amssymb}
\usepackage{balance}
\usepackage{booktabs}
\usepackage{graphicx}
\usepackage{multirow}
\usepackage{needspace}
\usepackage{microtype}
\usepackage{bold-extra}
\usepackage{subfigure}
\usepackage{wrapfig}


% constants
\newcommand{\Title}{Execution Blueprint: An Effective Visual Support to Monitor Software Execution}
\newcommand{\TitleShort}{\Title}
\newcommand{\Authors}{Juan Pablo Sandoval Alcocer, Alexandre Bergel}
\newcommand{\AuthorsShort}{J.P. Sandoval Alcocer, A. Bergel}

% references
\usepackage[colorlinks]{hyperref}
\usepackage[all]{hypcap}
\setcounter{tocdepth}{2}
\hypersetup{
	colorlinks=true,
	urlcolor=black,
	linkcolor=black,
	citecolor=black,
	plainpages=false,
	bookmarksopen=true,
	pdfauthor={\Authors},
	pdftitle={\Title}}

\def\chapterautorefname{Chapter}
\def\appendixautorefname{Appendix}
\def\sectionautorefname{Section}
\def\subsectionautorefname{Section}
\def\figureautorefname{Figure}
\def\tableautorefname{Table}
\def\listingautorefname{Listing}

% source code
\usepackage{xcolor}
\usepackage{textcomp}
\usepackage{listings}
\definecolor{source}{gray}{0.9}
\lstset{
	language={},
	% characters
	tabsize=3,
	upquote=true,
	escapechar={!},
	keepspaces=true,
	breaklines=true,
	alsoletter={\#:},
	breakautoindent=true,
	columns=fullflexible,
	showstringspaces=false,
	basicstyle=\footnotesize\sffamily,
	% background
	frame=single,
    framerule=0pt,
	backgroundcolor=\color{source},
	% numbering
	numbersep=5pt,
	numberstyle=\tiny,
	numberfirstline=true,
	% captioning
	captionpos=b,
	% formatting (html)
	moredelim=[is][\textbf]{<b>}{</b>},
	moredelim=[is][\textit]{<i>}{</i>},
	moredelim=[is][\color{red}\uwave]{<u>}{</u>},
	moredelim=[is][\color{red}\sout]{<del>}{</del>},
	moredelim=[is][\color{blue}\underline]{<ins>}{</ins>}}
\newcommand{\ct}{\lstinline[backgroundcolor=\color{white},basicstyle=\footnotesize\ttfamily]}
\newcommand{\lct}[1]{{\small\tt #1}}

% tikz
% \usepackage{tikz}
% \usetikzlibrary{matrix}
% \usetikzlibrary{arrows}
% \usetikzlibrary{external}
% \usetikzlibrary{positioning}
% \usetikzlibrary{shapes.multipart}
% 
% \tikzset{
% 	every picture/.style={semithick},
% 	every text node part/.style={align=center}}

% proof-reading
\usepackage{xcolor}
\usepackage[normalem]{ulem}
\newcommand{\ra}{$\rightarrow$}
\newcommand{\ugh}[1]{\textcolor{red}{\uwave{#1}}} % please rephrase
\newcommand{\ins}[1]{\textcolor{blue}{\uline{#1}}} % please insert
\newcommand{\del}[1]{\textcolor{red}{\sout{#1}}} % please delete
\newcommand{\chg}[2]{\textcolor{red}{\sout{#1}}{\ra}\textcolor{blue}{\uline{#2}}} % please change
\newcommand{\chk}[1]{\textcolor{ForestGreen}{#1}} % changed, please check

% comments \nb{label}{color}{text}
\newboolean{showcomments}
\setboolean{showcomments}{true}
\ifthenelse{\boolean{showcomments}}
	{\newcommand{\nb}[3]{
		{\colorbox{#2}{\bfseries\sffamily\scriptsize\textcolor{white}{#1}}}
		{\textcolor{#2}{\sf\small$\blacktriangleright$\textit{#3}$\blacktriangleleft$}}}
	 \newcommand{\version}{\emph{\scriptsize$-$Id$-$}}}
	{\newcommand{\nb}[2]{}
	 \newcommand{\version}{}}
\newcommand{\rev}[2]{\nb{Reviewer #1}{red}{#2}}
\newcommand{\ab}[1]{\nb{Alexandre}{blue}{#1}}
\newcommand{\vp}[1]{\nb{Vanessa}{orange}{#1}}
\newcommand{\jp}[1]{\nb{Juan Pablo}{green}{#1}}

% graphics: \fig{position}{percentage-width}{filename}{caption}
\DeclareGraphicsExtensions{.png,.jpg,.pdf,.eps,.gif}
\graphicspath{{figures/}}
\newcommand{\fig}[4]{
	\begin{figure}[#1]
		\centering
		\includegraphics[width=#2\textwidth]{#3}
		\caption{\label{fig:#3}#4}
	\end{figure}}

\newcommand{\largefig}[4]{
	\begin{figure*}[#1]
		\centering
		\includegraphics[width=#2\textwidth]{#3}
		\caption{\label{fig:#3}#4}
	\end{figure*}}
	
\newcommand{\wrapfig}[5]{	
\begin{wrapfigure}{#1}{#2\textwidth}
  \begin{center}
    \includegraphics[width=#3\textwidth]{#4}
  \end{center}
  \caption{\label{fig:#4}#5}
\end{wrapfigure}}

% abbreviations
\newcommand{\ie}{\emph{i.e.,}\xspace}
\newcommand{\eg}{\emph{e.g.,}\xspace}
\newcommand{\etc}{\emph{etc.}\xspace}
\newcommand{\etal}{\emph{et al.}\xspace}

% lists
\newenvironment{bullets}[0]
	{\begin{itemize}}
	{\end{itemize}}

\newcommand{\seclabel}[1]{\label{sec:#1}}
\newcommand{\figlabel}[1]{\label{fig:#1}}
\newcommand{\figref}[1]{Figure~\ref{fig:#1}}
\newcommand{\secref}[1]{Section~\ref{sec:#1}}

%%%%%%%%%%%%%%%%%%%%%%%%%%%%%% adding smalltalk listing %%%%%%%%%%%%%%%%%%%5
\lstdefinelanguage{Smalltalk}{
  morekeywords={true,false,self,super,nil},
  sensitive=true,
  morecomment=[s]{"}{"},
  morestring=[d]',
  style=SmalltalkStyle
}
\lstdefinestyle{SmalltalkStyle}{
  literate={:=}{{$\gets$}}1{^}{{$\uparrow$}}1
} 
%%%%%%%%%%%%%%%%%%%%%%%%%%%%%% adding smalltalk listing %%%%%%%%%%%%%%%%%%%5

%Specialized macros
\newcommand{\hapao}{Hapao\xspace}
\newcommand{\Hapao}{Hapao\xspace}
\pagenumbering{arabic}

\begin{document}

\title{\Title}
%\titlerunning{\TitleShort}

\author{\Authors\\[3mm]
Department of Computer Science (DCC)\\ University of Chile, Santiago, Chile\\[1 ex]
} 
%\authorrunning{\AuthorsShort}

\maketitle

\emph{This paper makes use of colored figures. Though colors are not mandatory for full understanding, we recommend  the use of a colored printout.}

\begin{abstract}
%	What's the problem.
%	Why is the problem a problem?
%	What's the surprising idea?
%	What's the consequence?

Understanding the behavior of software is one essential task of the software life cycle, especially when maintenance activities have to be performed. One has to analyze the code, the documentation or any other device before making any changes therefore it becomes a difficult task and time consuming.

Our goal is to show a visualization that represents the interaction of objects involved in the execution which can help developers gain a better understanding of the software execution. The large number of objects involved in execution represents a major challenge.

Chaski is a tool to address this problem of understanding by visualizing the execution of an object oriented program. Execution blueprint shows the interaction between objects through messages exchanged between them, in the order they occur. \ab{Say more about the visualization}. This visualization helps to: understand the current behavior of the software, find candidates for refactoring and optimization, show dependencies and relations, among others.


\end{abstract}

%: % % % % % % % % % % % % % % % % % % % % % % % % % % % % % % % % %

\section{Introduction}\seclabel{introduction}

The use of highly reflective dynamic languages can make seemingly simple questions(how 2 or more packages interact? What kinds are dependent to a certain class? When exactly this method is invoked?)difficult to answer if we just analyze the source code, documentation or any other appliance design.

To answer these questions among others and to improve the process of understanding, it is necessary to make a dynamic analysis, or analysis of data retrieved from the execution of a program.

In this context, we propose a visualization of the execution of an object oriented program, showing the interaction between objects through messages exchanged between them, in the order they occur. While understanding the overall behavior of the system is ideal, this visualization was intended to be used by objectives, in other words, to analyze portions of the application that really interest us to achieve an objective.

The purpose of this visualization is to help: understand the current behavior of the software, find candidates for refactoring and optimization, show dependencies and relations, among others.

%: % % % % % % % % % % % % % % % % % % % % % % % % % % % % % % % % %
\section{Execution Blueprint}\seclabel{executionBlueprint}

\subsection{In a nutshell}

Execution Blueprint is a visualization that represent\ins{s} the timeline of an application execution. It is obtained after to execute the application according to an execution scenario. An execution scenario being an instance of one or more use cases.

\largefig{}{1.0}{MondrianDependence}{How packages/class interact in the execution? (mondrian)}

\chg{In Execution Blueprint we can}{Execution blueprint} distinguish\ins{es} two types of components; the classifiers: package, class, object which are identified by the colors blue, red, yellow  respectively; and the messages, which are shown with a gray color \del{by default}. Depending on what we want to \chg{analyze in the visualization}{use the visualization for}, we can set different colors to the messages based on a criteria. In the case of \figref{MondrianDependence}, the magenta color was established for "initialize" \ab{You should write ``initialize'' and not "initialize". Look at the quotes.} messages\ab{So what we use the visualization for? Explain a bit more this example}.\ab{Maybe we could have a short description of the blueprint. Can you draw a small visualization and what is the source code that produces it? Something small, a bit like in http://hapao.dcc.uchile.cl:8090/pier/system/components/menu/Test-Blueprint} 

The visualization shows the interaction between objects, classes and packages (classifiers) through messages sent or received by them, where the messages are vertically aligned with the component receiving the message and placed horizontally in the order in which these messages \chg{were}{are} \ab{use the present tense, and not the past} received\ab{Sentence too long, please shorten it}.

The edges are an important \ugh{factor}\ab{an edge is not a factor, it is a graphical element} fact in allowing the visualization to link two messages, in the case of Figure 1 the axes represent invocation between methods, being the message at the left the one that invokes to the message at the right\ab{Sentence too long}.

\ab{I think the visualization is more about identifying execution patterns. What do we see in Figure 1?}  

All this allows to \ab{never use ``allow to''. You can either use ``allow someone to do'' or ``allow for doing''} determine which object \chg{is the one that receives more messages}{receives the most messages}, which object needs to communicate with objects from other packages, which objects are instantiated at runtime (magenta messages) and how many instances of a particular class participate in the execution.

%=========
\subsection{Layout}

The essence of ``layout'' are the messages and the chronological order in which messages are receive\ins{d}. \chg{And provide classifiers in order to group the messages}{The layout orders the messages into group}. As seen in the example in Figure 1. the messages are grouped first by class package and finally by object, note that some messages are not directly associated with an object but with a class, meaning that the receiver of the message is the same class (a class method), this is illustrated in \figref{MondrianDependence} (\del{the letter} A). \ab{I do not understand this. I would like to understand what is the cloud of points near A. What does this mean?}

This classification or grouping of \ugh{posts} may vary depending on what one may want to analyze. For example in \figref{ChaskiPreview8} messages are just grouped simply by object, in order to analyze the dependency between objects.

While the messages are horizontally sorted in a chronological order, the other components (package, class, object) are sorted vertically by number of messages that has been sent, in this sense the component that sends more messages are located above.

The relationships between messages is an important factor(edges) in the visualization.We consider two types of relationships between messages.

\begin{itemize}
\item Dependency \chg{.-}{--} where a message is related to the message from which it was invoked \ins{(}\figref{MondrianDependence}\ins{)}.
\item Sequence.- where a message is simply associated with another in the order they were executed \figref{ChaskiPreview8}.
\end{itemize}
%=========
\subsection{Patterns}

Execution Blueprint allows classifying \ugh{repetitive}\ab{not only repetitive I have the impression} behavior(patterns) to help the developer to have a better understanding of the behavior of software and to reduce the analysis time of the visualization\ab{Yeah, this is what most of the visualization wants to do. We want to do better actually: identifying some abnormal or singular execution pattern in order to improve the program}.

\begin{itemize}
\item Identity Behavior.- When two or more sections of the visualization have identical behavior(same sequence of messages) and there is a perfect match between objects and messages. Which implies that the messages of each section are transmitted and received by the same objects. \ab{Give example} \ab{What is the possible improvement you can do?}
\item Class Identity Behavior.- When two or more sections of the visualization have the same sequence of messages but the messages of each section were sent and received by different instances from the same class.(\figref{ChaskiPreview11}). \ab{What is the possible improvement you can do?}
\end{itemize}

That is the case of \figref{ChaskiPreview8}, in which the behavior of the execution of a Test Suite is analyzed. Note that one of the important features of a suite of tests is that they should be independent of each other: the failure of one test should not cause an avalanche of failures of other tests that depend upon it, nor should the order in which the tests are run matter. Performing setUp before each test and tearDown afterwards helps to reinforce this independence.


\largefig{}{1.0}{ChaskiPreview11}{Finding patterns in the execution (Glamour)}
\largefig{}{1.0}{ChaskiPreview8}{Visualizing a Test Suite execution (RBSmallDictionaryTest buildSuite run)}
%: % % % % % % % % % % % % % % % % % % % % % % % % % % % % % % % % %
\section{Case Study}\seclabel{case study}

\subsection{Mondrian}

We will use Mondrian to demonstrate how ``Execution BluePrint'' visualization is useful to understand the current behavior of the software and find candidates for refactoring and optimization. To which we will address the visualization for the following code execution \figref{Mondrian1}:
\begin{lstlisting}[language=Smalltalk]
	...
	view := MOViewRenderer title: 'Mondrian View Renderer'.
	view nodes: subclasses.
	view edgesFrom: #superclass.
	view treeLayout.
	...
\end{lstlisting}

\subsection{Execution anomalies}

In this section we will examine \figref{Mondrian1}, which groups messages only for package and class. This type of grouping is useful to detect the pattern "Class Behavior Identity" more easily .

In the display you can see the existence of behavior repeated in several places, we  will next analyze the sequence of messages in this box. It was found the method responsible for causing this behavior:

\begin{lstlisting}[language=Smalltalk]
MOShape class>>defaultNodeShapeClass
		"Return the default node shape class"
		| classes |
		classes := (self allSubclasses select: #isDefaultNodeShape).
		classes ifEmpty: [ self error: 'No default shape class' ].
		^ classes anyOne.
\end{lstlisting} 

The method defaultNodeShapeClass has the function to search through a MOShape subclasses one class established by default. Through the method "isDefaultNodeShape" shown below.

\begin{lstlisting}[language=Smalltalk]
MORectangleShape class>>isDefaultNodeShape
	^ true
\end{lstlisting} 

\ab{What is the problem example. I know you said it with the pattern, but you have to say that this problem is serious, big and worth addressing it.}

\largefig{}{1.0}{Mondrian1}{Before Improvement}

\subsection{Improvement}

Based on the above analysis we can deduce that it is not necessary to search twice because it could not be changed or set by default to a subclass of MOShape at runtime. That is, the result of the search will always be the same during the execution.\ab{This is too rough and harsh. I cannot really understand this. Explain a bit further}

Therefore, the code was improved using Cache (\figref{Mondrian1}):

\begin{lstlisting}[language=Smalltalk]
MOShape class>>defaultNodeShapeClass
	"Return the default node shape class"
	| classes |
	"Cache"
	DefaultNodeShapeClass ifNotNil:[^DefaultNodeShapeClass].
	classes := (self allSubclasses select: #isDefaultNodeShape).
	classes ifEmpty: [ self error: 'No default shape class' ].
	"Cache"
	^ DefaultNodeShapeClass := classes anyOne.
\end{lstlisting} 

After we made these changes, the execution is shown as follows in Figure 5.
\largefig{}{1.0}{Mondrian2}{After Improvement}

%And we established a method for removing the Cache:
%\begin{lstlisting}[language=Smalltalk]
%MOShape class>>resetCache
%	DefaultNodeShapeClass :=nil.
%\end{lstlisting} 
%Since it is possible for someone to set some other default MOShape before running the application, the cache is removed before Mondrian carries out any %%visualization:
%\begin{lstlisting}[language=Smalltalk]
%MOViewRenderer>>initialize
%	MOShape resetCache.
%	...
%\end{lstlisting}

%: % % % % % % % % % % % % % % % % % % % % % % % % % % % % % % % % %
\section{Managing the scalability}\seclabel{scalability}

The large number of objects and messages involved in an object-oriented program execution makes the visualization extremely large and difficult to analyse.

In order to reduce the visualization size is recommended to group the messages simply by class and package. In addition we can use colors in messages can help to a better analisis.

Being able to find repetitive behavior,it is possible to have a reduced version of litle set of message sequence that represent all execution. This would allow greatly reduce the visualializacion, in the future.

%: % % % % % % % % % % % % % % % % % % % % % % % % % % % % % % % % %
\section{Discussion}\seclabel{discussion}

The large-scale sequence of messages makes the screen resolution not enough to show a lot of information,causing the emergence of scrolls that make the graphics difficult to understand. Group the messages only by class and packages reduce the height of a visualization. However the big number of messages remains a problem.


%: % % % % % % % % % % % % % % % % % % % % % % % % % % % % % % % % %
\section{Implementation}\seclabel{implementation}

\subsection{Chaski}

Chaski is a tool that implement the layout of Execution Blueprint in Pharo.The following code is a chaski script that visualize the \figref{MondrianDependence}.
 
\begin{lstlisting}[language=Smalltalk]
|view|
view := MOViewRenderer new. 
Chaski profile: [ 	
	view nodes: (1 to: 4). 
] inPackagesMatching: 'Mondrian-*'.
Chaski config ifMessage: [ :aChaskiMessage | 
	aChaskiMessage method selector ='initialize'.
	] fillColor: Color magenta.
Chaski visualize.
\end{lstlisting}

Chaski allow apply filter in the visualizations components and set colors to messages in base a criteria. In the case of last code we apply a filter in order to display only packages that start with "Mondrian-" and fill "initialize" messages with magenta color.

\fig{}{0.3}{Architecture}{Chaski}

As shown in the \figref{Architecture}, Chaski was created as an especification of Spy coupled with Mondrian to render profiled information.

\fig{}{0.5}{ObjectModel}{Structure of Chaski}

The core classes for Chaski are depicted in \figref{ObjectModel}. Note that there are two additional classes to the basic structure of Spy, they are ChaskiObject and ChaskiMessage. This is because "Execution Blueprint" makes an analysis beyond class or method.

Chaski classify messages by packages, class and object by default. And provide options to group messages with different criteria such as \figref{Mondrian1} that only group messages by packages and class. The code below remove object classifier from Chaski default configuration.
\begin{lstlisting}[language=Smalltalk]
...
Chaski config displayObjectsThat:[ :aChaskiObject | false].
Chaski visualize.
\end{lstlisting}

Also is posible classify messages only by object, with the code below. Where we need remove package and class classifiers form Chaski default configuration (\figref{ChaskiPreview8}).
\begin{lstlisting}[language=Smalltalk]
...
Chaski config displayPackagesThat:[:aChaskiPackage | false].
Chaski config displayClassesThat:[ :aChaskiClass | false].
Chaski visualize.
\end{lstlisting}

%: % % % % % % % % % % % % % % % % % % % % % % % % % % % % % % % % %
\section{Related work}\seclabel{relatedwork}

Research into trace visualization has resulted in various techniques and tools.


%: % % % % % % % % % % % % % % % % % % % % % % % % % % % % % % % % %

\section{Conclusion}\seclabel{conclusion}

%\paragraph{Acknowledgment}
% % % % % % % % % % % % % % % % % % % % % % % % % % % % % % % % % %

\bibliographystyle{plain}
\bibliography{scg}

\end{document}

