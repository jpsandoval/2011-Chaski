% Template article for preprint document class `elsart'
% SP 2006/04/26

\documentclass{elsart}

% Use the option doublespacing or reviewcopy to obtain double line spacing
% \documentclass[doublespacing]{elsart}

% if you use PostScript figures in your article
% use the graphics package for simple commands
% \usepackage{graphics}
% or use the graphicx package for more complicated commands
% \usepackage{graphicx}
% or use the epsfig package if you prefer to use the old commands
% \usepackage{epsfig}

% The amssymb package provides various useful mathematical symbols
\usepackage{amssymb}

% The lineno packages adds line numbers. Start line numbering with
% \begin{linenumbers}, end it with \end{linenumbers}. Or switch it on
% for the whole article with \linenumbers.
% \usepackage{lineno}

% \linenumbers
\begin{document}

\begin{frontmatter}

% Title, authors and addresses

% use the thanksref command within \title, \author or \address for footnotes;
% use the corauthref command within \author for corresponding author footnotes;
% use the ead command for the email address,
% and the form \ead[url] for the home page:
% \title{Title\thanksref{label1}}
% \thanks[label1]{}
% \author{Name\corauthref{cor1}\thanksref{label2}}
% \ead{email address}
% \ead[url]{home page}
% \thanks[label2]{}
% \corauth[cor1]{}
% \address{Address\thanksref{label3}}
% \thanks[label3]{}

\title{Chaski}

% use optional labels to link authors explicitly to addresses:
% \author[label1,label2]{}
% \address[label1]{}
% \address[label2]{}

\author{Juan Pablo}

\address{}

\begin{abstract}
% Text of abstract
\end{abstract}

\begin{keyword}
% keywords here, in the form: keyword \sep keyword

% PACS codes here, in the form: \PACS code \sep code
\PACS 
\end{keyword}
\end{frontmatter}

% main text
\section{}

The main difficulty of making a dynamic analysis is the large amount of data collected from the execution of code, so as performance and scalability problems.

One of the biggest problems in software maintenance is to understand the behavior of the software, dynamic analysis is necessary to perform this task.

It is difficult to navigate the vast(large) amount of information obtained from the code execution, always trying not to get lost in the visualization. (Browser)


Layout Chaski

The essence of chaski are the messages and the chronological order in which these (messages) occur.

The large-scale sequence of messages makes the screen resolution not enough to show a lot of information,causing the emergence of scrolls that make the graphics difficult to understand.

The layout of chaski allows the user to group the messages by different criteria, such as per package, class, object, method with blue, red, yellow and green colors, respectively, or a combination of these but respecting the hierarchy.

Detecting important messages.

To help mark or highlight the messages you want to analyze, chaski provides a simple mechanism of colored rules of messages, it allows to color a message given a particular criterion. If you have multiple criteria and, in the case that one or more messages meet more than one criterion, these will take the color of the last criterion.

Important message filtering.

So like a lot of objects are created when executing an object-oriented program,  so much larger the number of messages that are sent between objects, making the visualization extremely large. The filters in chaski try to display messages that meet a certain condition allowing a smaller view and most importantly, making  possible to analyze only the messages that interest us.

\label{}

% The Appendices part is started with the command \appendix;
% appendix sections are then done as normal sections
% \appendix

% \section{}
% \label{}

\begin{thebibliography}{00}

% \bibitem{label}
% Text of bibliographic item

% notes:
% \bibitem{label} \note

% subbibitems:
% \begin{subbibitems}{label}
% \bibitem{label1}
% \bibitem{label2}
% If there is a note, it should come last:
% \bibitem{label3} \note
% \end{subbibitems}

\bibitem{}

\end{thebibliography}

\end{document}

